\documentclass[11pt]{beamer}
\usetheme{Rochester}
\usecolortheme{seagull}
\usepackage[utf8]{inputenc}
\usepackage[german]{babel}
\usepackage[T1]{fontenc}
\usepackage{amsmath}
\usepackage{amsfonts}
\usepackage{amssymb}
\usepackage{textpos}

% add logo to the sections
\addtobeamertemplate{frametitle}{}{%
\begin{textblock*}{100mm}(.6\textwidth,-1.2cm)
\includegraphics[width=0.5\linewidth]{./logo_inf_fak.png}
\end{textblock*}}


\setbeamertemplate{footline}[frame number]

\author{Größler, Hofmann, Zettwitz, Sopauschke, Sturm, Solovjova}
\title{A visualization technique for \\  hierarchical edge bundles}
\date{21.Januar 2016} 
%\setbeamercovered{transparent}  
%\subject{}


\begin{document}

\begin{frame}
\titlepage
\end{frame}

\begin{frame}
\frametitle{Inhalt} 
\tableofcontents
\end{frame}


\section{Ziel des Projekts}
\begin{frame}
\frametitle{Ziel des Projekts}
\begin{itemize} 
\item hierarchische Graphdaten (Baum)
\item Radiales Layout der Graphdaten
\item B ündelung der Graphdaten zu B-Splines
\item RT/interaktive Visulaisierung
\end{itemize}
\end{frame}



\section{Compound Graph}
\begin{frame}[allowframebreaks]
\frametitle{Compound Graph}

Compound Graph components:
\begin{itemize} 
\item Knoten (parent, children)                                                                                                                       
\item Level, Link, Position(x,y)
\item Anzahl an Kindern auf allen Leveln
\end{itemize}

\framebreak
\begin{itemize} 
\item Pfad nur von Blatt zu Blatt
\item Shortest path: von beiden Endpunkten aufsteigen \\ bis gemeinsamer Knoten erreicht wurde
\item Speichere jeden besuchten Knoten (später für Splines benötigt)
\end{itemize}

%Übersichtsbild der Klassen einfügen

\end{frame}



\section{Splines}
\begin{frame}
\frametitle{Splines}
\begin{itemize}
\item dummy splines allgemein
\end{itemize}

\end{frame}



\subsection{B-Splines}
\begin{frame}
\frametitle{B-Splines}

\begin{itemize}
\item dummy
\end{itemize}

\end{frame}



\section{Technnisches System}
\begin{frame}
\frametitle{Technisches System}
\begin{itemize}
\item dummy cross platform, OGL, GCC, GLUT, GLEW, CMAKE...
\end{itemize}

\end{frame}



\section{Demo}
\begin{frame}
\frametitle{Demo}
Es folgt eine kurze Demonstration unserer Ergebnisse

\end{frame}


\section{Nutzen}
\begin{frame}
\frametitle{Nutzen}
\begin{itemize}
\item Datenexplorariton...
\end{itemize}

\end{frame}


\end{document}